%%% sat09-cfp.tex --- 

% Original author: Joao Marques-Silva
% Adapted by Oliver Kullmann, 22.10.2008.

%%%%##########################################################################
%%%% Configurations
%%%%
\documentclass[10pt]{article}

\usepackage[usenames]{color}
\usepackage{url}
\usepackage{hyperref}

\usepackage{a4wide}


%% Redefine the page layout
%%
\addtolength{\textheight}{60pt}
\addtolength{\topmargin}{-60pt}
\addtolength{\textwidth}{100pt}
\addtolength{\oddsidemargin}{-50pt}
\addtolength{\evensidemargin}{-50pt}
\setlength{\marginparwidth}{0pt}

\def \baselinestretch{0.95}

\newcommand{\bthlight}[1]{{\color[rgb]{0.2,0.2,0.5}#1}}
\newcommand{\bhlight}[1]{{\color[rgb]{0.2,0.2,0.6}#1}}
\newcommand{\ghlight}[1]{{\color[rgb]{0.1,0.4,0.1}#1}}

\definecolor{URLGreen}{rgb}{0.1,0.5,0.1} 

\hypersetup{colorlinks=true,pdfborder={0 0 0},urlbordercolor={0 0 0},
urlcolor=URLGreen}
%linkcolor={0.1 0.4 0.1},filecolor={0.1 0.4 0.1}


%%%%##########################################################################
%%%% The main document
%%%%
\begin{document}

\pagestyle{empty}

\begin{center}
  {\bf
    \bthlight{{\huge SAT 2009}} \\[0.3cm]
    {\Large Call for Papers} \\[0.15cm]
    \bthlight{
      {\LARGE 12th International Conference on} \\[0.2cm]
      {\LARGE Theory and Applications of Satisfiability Testing} \\[0.15cm]
      {\Large June 30 - July 3, 2009, Swansea, Wales, United Kingdom}
    } \\[0.15cm]
    {\large \url{http://www.cs.swan.ac.uk/~csoliver/SAT2009}}
  }
\end{center}

\vspace*{-0.1cm}
%
\begin{minipage}[t]{7.5cm}
{\large {\bf \bhlight{Conference Chair}}} \\[0.05cm]
{\small
  Oliver Kullmann, Swansea University, Swansea, UK\\[0.3cm]
}
{\large {\bf \bhlight{Invited Speakers}}} \\%[0.1cm]
{\small
  Robert Nieuwenhuis, Tech.~Univ.~Catalonia, Spain\\
  Moshe Vardi, Rice University, U.S.A.\\[0.3cm]
}
{\large {\bf \bhlight{Important Dates}}} \\[0.05cm]
{\small
  February 20, {\em Abstract Submission} \\
  February 27, {\em Paper Submission} \\
  March 22, {\em Author Notification} \\
  March 29, {\em Final Version} \\[0.3cm]
}
{\large {\bf \bhlight{Technical Program Committee}}} \\[0.05cm]
{\small
  {\bf TBA}
%   Fahiem Bacchus, University of Toronto, Canada \\
%   Paul Beame, University of Washington, USA \\
%   Armin       Biere, Johannes Kepler Univ., Austria \\
%   Adnan Darwiche, UCLA, USA \\
%   Leonardo de Moura, Microsoft Research, USA \\
%   Niklas \'{E}en, Cadence Design Systems, USA \\
%   John Franco, University of Cincinnati,  USA\\
%   Ian Gent, University of ~St.~Andrews, UK \\
%   Enrico Giunchiglia, Univ.~di Genova, Italy \\
%   Carla Gomes, Cornell University, USA \\
%   Aarti Gupta, NEC Research Labs, USA \\
%   Ziyad Hanna, Intel Corp., Israel \\
%   Edward A.~Hirsch, Steklov Inst.~Mathematics, Russia \\
%   Joonyoung Kim, Intel Corp., USA \\
%   Hans Kleine-B\"{u}ning, Univ.~Paderborn, Germany\\
%   James Kukula, Synopsys ATG, USA \\
%   Oliver Kullmann, University of Wales Swansea, UK \\
%   Daniel Le Berre, Universit\'{e} d'Artois, France \\
%   Chu-Min Li, Universit\'{e} de Picardie, France \\
%   Ines Lynce, Tech.~Univ.~Lisbon, Portugal \\
%   Panagiotis Manolios, Georgia Inst.~Technology, USA \\
%   Vasco Manquinho, Tech.~Univ.~Lisbon, Portugal \\
%   Slawomir Pilarski, Magma DA, USA \\
%   Steve Prestwich, University College Cork, Ireland \\
%   Roberto Sebastiani, Univ.~Trento, Italy \\
%   Hossein Sheini, CMU, USA \\
%   Laurent Simon, Universit\'{e} Paris Sud, France \\
%   Ewald Speckenmeyer, Universit\"{a}t~K\"{o}ln, Germany \\
%   Ofer Strichman, Technion, Israel \\
%   Stefan Szeider, Durham University, UK \\
%   Armando Tacchella, Univ.~di Genova, Italy \\
%   Allen Van Gelder, UC Santa Cruz, USA \\
%   Hans van Maaren, Tech.~Univ.~Delft, Netherlands \\
%   Toby Walsh, National ICT, Australia \\
%   Lintao Zhang, Microsoft Research, USA
}
\end{minipage}
\hspace*{0.35cm}
\begin{minipage}[t]{9.25cm}
  \begin{minipage}[t]{9.25cm}
    \vspace*{0.15cm}
  {\small
  The International Conference on Theory and Applications of
  Satisfiability Testing (\href{http://www.satisfiability.org}{SAT})
  is the primary annual meeting for researchers studying the
  propositional satisfiability
  problem. \href{http://cs-svr1.swan.ac.uk/~csoliver/SAT2009}{SAT
    2009} is the twelfth SAT conference. SAT 2009 features the
  \href{http://www.satcompetition.org/2009}{SAT competition}, the
  \href{http://www.cril.univ-artois.fr/PB09}{Pseudo-Boolean evaluation}, and the
  \href{http://www.maxsat07.udl.es}{MAX-SAT evaluation}. \\[0.1cm]
%
  Many hard combinatorial problems can be encoded into
  SAT. Therefore improvements on heuristics on the practical side, as
  well as theoretical insights into SAT, apply to a large range of
  real-world problems. More specifically, many important practical
  verification problems can be rephrased as SAT problems. This
  applies to verification problems in hardware and software. Thus SAT
  is becoming one of the most important core technologies to verify
  secure and dependable systems. The topics of the conference span
  practical and theoretical research on SAT and its applications and
  include but are not limited to proof systems, proof complexity,
  search algorithms, heuristics, analysis of algorithms, hard
  instances, randomised formulae, problem encodings, industrial
  applications, solvers, simplifiers, tools, case studies and
  empirical results. SAT is interpreted in a rather broad sense:
  besides propositional satisfiability, it includes the domain
  of quantified boolean formulae (QBF), constraints programming
  techniques (CSP) for word-level problems and their propositional
  encoding and particularly satisfiability modulo theories (SMT). \\[0.1cm]
%
  Submissions should contain original material and can either be
  regular research papers up to 14 pages or short papers up to 6
  pages. Double submissions including submissions as short and long
  papers will be rejected.  Submissions should use the 
  \href{http://www.springer.com/comp/lncs/Authors.html}{Springer LNCS}
  style. All appendices, tables, figures and the bibliography 
  must fit into the page limit. Submissions deviating from these
  requirements may be rejected without review. All accepted papers
  including short papers will be published in the proceedings of the
  conference. The conference proceedings will be published within
  the Springer LNCS series. The paper submission page is
  \url{http://www.easychair.org/conferences/?conf=sat2009}. Papers have to
  be submitted electronically as PDF files. Abstract submissions are due
  Februay 20, paper submissions February 27.
  }
  \end{minipage}

  \begin{minipage}[t]{9.25cm}
    \vspace*{0.35cm}
    \begin{minipage}[t]{4.5cm}
      \bthlight{{\bf SAT Competition}} \\
               {\small \url{www.satcompetition.org/2009}}\\[0.15cm]
               Daniel Le Berre\\
               Laurent Simon\\[0.15cm]
               Ewald Speckenmeyer\\
               Geoff Sutcliffe \\
               Lintao Zhang %\\ [0.15cm]
    \end{minipage}
    \hspace*{0.5cm}
    \begin{minipage}[t]{4.25cm}
      \bthlight{{\bf MAX-SAT Evaluation}}\\
               {\small \url{www.maxsat07.udl.es}}\\[0.15cm]
               Josep Argelich\\
               Chu-Min Li\\
               Felip Many\`{a}\\
               Jordi Planes %\\[0.15cm]
    \end{minipage}

    \begin{minipage}[t]{9.25cm}
      \vspace*{0.25cm}
      \begin{minipage}[t]{4.5cm}
        \bthlight{{\bf PB Evaluation}}\\
                 {\small \url{www.cril.univ-artois.fr/PB09}}\\[0.15cm]
                 Vasco Manquinho\\
                 Olivier Roussel %\\[0.15cm]
      \end{minipage}
    \end{minipage}
  \end{minipage}


\end{minipage}
%

\end{document}

%%%%
%%%%##########################################################################

